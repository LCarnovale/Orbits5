\documentclass{report}
\usepackage{fullpage}

\usepackage{physics}
\usepackage{tikz}
\usepackage{amsmath, graphicx, xspace}
\usepackage{amssymb, amsthm}
\usepackage{enumitem}
\usepackage{siunitx}
\usepackage{hyperref}

% \usepackage[backend=bibtex,style=verbose-trad1, citestyle=numeric]{biblatex}


\newcommand\xp{x\prime\xspace}
\newcommand\yp{y\prime\xspace}
\newcommand\vp{v\prime\xspace}

\newcommand{\bb}[1]{\mathbf{#1}}



\title{Orbits 5: A Graphical N-body Gravitational Simulator}
\author{Leonardo Carnovale}

\begin{document}
  \maketitle

  \tableofcontents
  \chapter{Simulator and Graphics API}
    \section{Relationship between simulator and API}
      The graphical display should not be a design specific to the
      operation of the simulator, so as to allow significant improvement
      and modification in future versions. This is to say that the
      simulator and the graphics API should operate independently, and
      communication or interaction should mainly be from the simulator to the camera,
      then to the API. The simulator will need to prioritise simulation, and spend all
      its time on this task. It should not try to optimize output for the API,
      this can be done within the API.

    \section{Function of the Simulator}
      The job of the simulator is simply to calculate the position of all objects
      after a time step interval. The main program will be frequently interacting with it.
      The simulator should have the following methods:
      \begin{itemize}
        \item Add an object to the simulation.
        \item Remove an object from the simulation.
        \item Change the default time step interval.
        \item Perform a step.
        \item Perform $n$ steps.
      \end{itemize}
      There are multiple ways to perform this simulation, and the current preference
      is the leapfrog method. Other methods should be available to be chosen at creation of
      the simulator object.

      The above methods will likely not need to have a return value. All information about the
      simulator such as the current time step, number of particles, or the information about a
      particular particle, should be available via additional `get' methods.
    \section{Function of the API}
      The API must have the following functionality as a minimum:
      \begin{itemize}
        \item Display a coloured pixel at a given point.
        \item Display a coloured circle or ellipse at a given location, filled or unfilled.
        \item Display a coloured regular or irregular polygon at a given location, filled or unfilled.
        \item Display coloured text at any position.
        \item Change the colour of the background/canvas.
        \item Clear the screen of all drawings instantly.
      \end{itemize}
      An extra `bonus' feature would be to save the image drawn to file.
      Graphics were achieved in previous versions using Python's Turtle library. This was very effective
      for small simple simulations. However, when displaying flares on stares or when drawing lots of particles
      the performance drops rapidly. Ideally, an additional API such as OpenGL should be used.

      The picture to be drawn by the API will be provided by the camera. Like the simulator, the camera should
      only have to do the bare mininum work before handing over to the graphics API. The only information provided
      to the api from the camera will be:
      \begin{itemize}
        \item Type of shape.
        \item Screen position of shape.
        \item Shape parameters. (If required)
        \item Shape fill options. (If required)
      \end{itemize}
      Potentially, the brightness or light intensity of the object will also be given.

  \chapter{Simulation Algorithms}
    \section{Matrix Mechanics}
      Generally using a static n-dimensional array as a matrix will be
      faster than using a dynamic n-dimensional list. This is especially
      the case with Numpy's arrays compared to the native python lists.

      To make calculations we require the following arrays to be known at all times:
      \begin{itemize}
        \item $\bb P_N \in \mathbb{R}^{N \times 3} \to$ Array of position vectors.

        \item $\bb R \in \mathbb{R^N} \to $ Array of radii of particles.

        \item $\bb M \in \mathbb{R^N} \to $ Array of masses of particles.
      \end{itemize}
      Where $N$ is the number of particles. For a calculation such as standard
      gravitational attraction, we would perform the following particle for each particle.
      Here the individual particle we are simulating for is as position $\bb P$,
      with mass $M$ and radius $R$.
      \begin{align*}
        \intertext{Find the distance to each other particle:}
        \bb D_N &= \bb P_N - \{\bb P \} \\
        \intertext{(The braces indicate the subtraction is for each element in the matrix $\bb P_N$)}
        \intertext{The absolute distances are given by:}
        \bb D &= |\bb D_N| = \{|\bb x| \forall \bb x \in \bb D_N \}
        \intertext{For the force:}
        \bb F_N &=  G M \odot \bb M \odot (\bb D_N) \oslash (\bb D^3) \\
        \intertext{Where $\odot, \oslash$ are element wise multiply and divide respectively.
                  Also note that cubic power at the right is element wise aswell.}
        \intertext{Now we sum the forces to get the net force:}
        \bb F &= \sum_{n = 0}^{N} \bb F_{N_n} \\
        \intertext{The acceleration is simply derived from $\bb F = m \bb a$, however it
                  would be fine to simply remove the mass of the particle at the start
                  in this case to directly calculate acceleration instead of force.}
        \intertext{We also want to check for collisions. Lets consider a vector $\bb A$
                  containing the `altitude' of each particle, ie the surface-surface distance.}
        \bb A &= \bb D - \bb R - \{R\} \\
        \bb B_\mathrm{collision} &= \{\bb A < 0\}
        \intertext{Where $\bb B$ is a boolean vector where a value of True indicates that
                    our current particle is colliding with the particle at that index.}
      \end{align*}











\end{document}
